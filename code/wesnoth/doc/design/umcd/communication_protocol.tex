\chapter{Communication protocol}
\label{chapter:communication_protocol}


\section{Server configuration}
\label{section:communication_protocol:server}

On the server there are several configuration files:
\begin{itemize}
\item
	The main configuration file, umcd.cfg, this file is stored in the
	data directory, which is a parameter for the programme. This file is
	further discussed in \cref{section:communication_protocol:server:umcd.cfg}.

\item
	In this data directory there is a subdirectory per addon. These
	directories have the following files:
	\begin{description}
	\item[umc.cfg]
		The config file for the umc, this file is based on the client's
		pbl file. This file is further discussed in
		\cref{section:communication_protocol:server:umc.cfg}.

	\item[umc.gz]
		The gzipped umc.

	\item[umc-\emph{lang}.cfg]
		The configuration file for the translation of the umc. The
		\emph{lang} is replace with the locale of the translation.
		This file is further discussed in
		\cref{section:communication_protocol:server:umc-lang.cfg}.

	\item[umc-\emph{lang}.gz]
		The gzipped mo-file for the translation. Note if the
		compression-ratio for the mo-file is bad, the raw mo-file will be
		stored instead. The final decision in the matter will be specified
		later.

	\end{description}

\end{itemize}


\subsection{umcd.cfg}
\label{section:communication_protocol:server:umcd.cfg}

This file contains the settings of the umcd.

\begin{lstlisting}
[umcd_configuration]
	port = PORT
	threads = THREADS

[/umcd_configurattion]
\end{lstlisting}

The keys have the following meaning:
\begin{description}
\item[PORT]
	The TCP/IP port to listen to for incoming requests.

\item[THREADS]
	The number of listeners threads to start. A value of $0$ means start as
	many threads as there are detected hardware cores\footnote{Whether
	hyper-threading and Co. count as separate cores is unspecified.}.

\end{description}


\subsection{umc.cfg}
\label{section:communication_protocol:server:umc.cfg}

This file contains the configuration of the umc on the server. The
combination of this file together with the various
\mbox{umc-\emph{lang}.cfg} files form the master information regarding an
umc. All other files that specify the umc or its wire-protocol get their
information from these files. Therefore other sections will refer to this
section.

The file is specified like:
\begin{lstlisting}
[umc]
	id = ID
	name = NAME
	icon = ICON
	type = TYPE
	description = DESCRIPTION
	version = VERSION
	size = SIZE
	authors = AUTHORS
	email = EMAIL
	uploader_ip_address = UPLOADER_IP_ADDRESS
	timestamp = TIMESTAMP
	downloads = DOWNLOADS
	uploads = UPLOADS
	password = PASSWORD
	language = LANGUAGE
	translatable = TRANSLATABLE
	[dependencies]?
		[dependency]+
			id = ID
			version? = VERSION_MASK
		[/dependency]
	[/dependencies]
[/umc]
\end{lstlisting}

The keys have the following meaning:
\begin{description}
\item[ID]
	The unique id of the content. When uploading this value determines
	whether an umc is the same umc.

\item[NAME]
	The name of the content. This string is not unique. This allows two
	versions of the UMC on the server. This can be used to allow a new beta
	next to a stable, as long as the ID differs.

\item[ICON]
	The icon of the umc, which is shown in the addon-list on the server.
	TODO it is not yet specified how the icon is transferred to and from the
	client.

\item[TYPE]
	The type of the addon. At the moment the following values are allowed:
	\begin{description}
	\item[CAMPAIGN\_SP] A single player campaign.
	\item[CAMPAIGN\_MP] A multi-player player campaign.
	\item[ERA] A multi-player era.
	\item[MAP] A multi-player map.
	\item[MAP\_PACK] A multi-player map pack.
	\item[RPG\_SP]
		A RPG like single player game. (This also includes builder games in
		the style of settlers.)

	\item[RPG\_MP]
		A RPG like multi-player game. (This also includes builder games in
		the style of settlers.)

	\item[OTHER] Everything that doesn't fit in one of the above.
	\end{description}

\item[DESCRIPTION]
	The description of the umc.

\item[VERSION]
	A string determining the version. The string should formatted like:
	`x.y.z' or `x.y.z-q'. When doing so the version can be parsed as version
	number. The `-q' part is sorted alphabetically and is an empty string if
	omitted, thus sorted first.

\item[SIZE]
	The size of the umc in bytes packed on the server.

\item[AUTHORS]
	A colon separated list of authors of the umc. When there is only one
	author no colon is required.

\item[EMAIL]
	The email address of the primary author or a mailinglist. When there are
	problems with the umc the Wesnoth team can use this address to contact
	the authors, so make sure it's valid and doesn't need registration.
	The address is not used in other ways nor shared with third-parties.

\item[UPLOADER\_IP\_ADDRESS]
	The ip address of the last uploader of the umc.

\item[TIMESTAMP]
	The timestamp of the last upload.

\item[DOWNLOADS]
	The number of times the UMC is downloaded.

\item[UPLOADS]
	The number of time a new version of the UMC is uploaded.

\item[PASSWORD]
	The password required to upload a new version of an umc or remove an umc
	from the server.

\item[LANGUAGE]
	The language id of the umc. If omitted American English is assumed.
	This allows umc authors who are not able to create content in English to
	still create content. However it's still recommended to try to write the
	content in American English. (TODO determine whether omitted on the
	server or only during the initial upload, if the latter the server
	should set it if omitted during the upload.)

\item[TRANSLATABLE]
	This flag causes the umc to be synchronised with Wescamp. It's advisable
	to only set the flags when the strings are somewhat stable. This flag
	can only be set if the language is American English.
\end{description}

The [dependencies] lists the dependencies of the UMC. This allows the client
to download all UMC required to use this UMC. The fields mean:
\begin{description}
\item[ID]
	The ID of the UMC to depend upon.

\item[VERSION\_MASK]
	This mask is used to validate a version used. The format is:
	\begin{description}
	\item[$=$VERSION]
		The version number of the dependency exactly match the VERSION.

	\item[$>=$VERSION]
		The version number of the dependency must be greater or equal to
		VERSION.

	\item[$>$VERSION]
		The version number of the dependency must be greater than VERSION.

	\item[$<$VERSION]
		The version number of the dependency must be less than VERSION
	\item[$<=$VERSION]
		The version number of the dependency must be less than or equal to
		VERSION

	\end{description}
	Several values can be used in a mask. They are separated by a space. The
	masks are and'ed and the user is responsible for setting a sane value;
	e.g. `$<$1.0.0 $>$1.0.0' can't match a version.

	If this field is omitted, all versions are allowed.

\end{description}


\subsection{umc-\emph{lang}.cfg}
\label{section:communication_protocol:server:umc-lang.cfg}.

The configuration file for the translation of the umc. The
\emph{lang} is replace with the locale of the translation.

\begin{lstlisting}
	[language]
		id = LANGUAGE
		timestamp = TIMESTAMP
		translated = TRANSLATED
		fuzzy = FUZZY
		untranslated = UNTRANSLATED
		[translated_umc_strings]
			name = NAME
			description = DESCRIPTION
		[/translated_umc_strings]
	[/language]
\end{lstlisting}



\section{UMC configuration}
\label{section:communication_protocol:umc}

The pbl file contains the information regarding the addon. Its contents are
stored on the system of the addon developer and on the server. The file
contains the following data:

\begin{lstlisting}
[umc_configuraton]
	id = ID
	name = NAME
	icon = ICON
	type = TYPE
	description = DESCRIPTION
	version = VERSION
	authors = AUTHORS
	email = EMAIL
	password? = PASSWORD
	language? = LANGUAGE
	translatable = TRANSLATABLE
	[dependencies]?
		[dependency]+
			id = ID
			version? = VERSION_MASK
		[/dependency]
	[/dependencies]
[/umc_configration]
\end{lstlisting}

The keys have the following meaning:
\begin{description}
\item[ID]
	See \cref{section:communication_protocol:server:umc.cfg}.

\item[NAME]
	See \cref{section:communication_protocol:server:umc.cfg}.

\item[ICON]
	See \cref{section:communication_protocol:server:umc.cfg}.

\item[TYPE]
	See \cref{section:communication_protocol:server:umc.cfg}.

\item[DESCRIPTION]
	See \cref{section:communication_protocol:server:umc.cfg}.

\item[VERSION]
	See \cref{section:communication_protocol:server:umc.cfg}.

\item[AUTHORS]
	See \cref{section:communication_protocol:server:umc.cfg}.

\item[EMAIL]
	See \cref{section:communication_protocol:server:umc.cfg}.

\item[PASSWORD]
	See \cref{section:communication_protocol:server:umc.cfg}.

	It may be omitted, during the initial upload. In that case the first
	upload results in sending back the password. It's stored in the pbl-file
	and the author doesn't need to remember it.

\item[LANGUAGE]
	See \cref{section:communication_protocol:server:umc.cfg}.

\item[TRANSLATABLE]
	See \cref{section:communication_protocol:server:umc.cfg}.

\end{description}

\section{Wire}
\label{section:communication_protocol:wire}

\section{Request campaign list}
\label{wire:request_campaign_list}

Request:
\begin{lstlisting}
	[request_umc_list]
	[/requst_umc_list]
\end{lstlisting}

Replay:
\begin{lstlisting}
	[umc_list]
		[umc]*
			id = ID
			name = NAME
			type = TYPE
			description = DESCRIPTION
			version = VERSION
			size = SIZE
			authors = AUTHORS
			timestamp = TIMESTAMP
			downloads = DOWNLOADS
			uploads = UPLOADS
			language = LANGUAGE
			[dependencies]?
				[dependency]+
					id = ID
					version? = VERSION_MASK
				[/dependency]
			[/dependencies]
			[languages]?
				[language]+
					id = LANGUAGE
					timestamp = TIMESTAMP
					translated = TRANSLATED
					fuzzy = FUZZY
					untranslated = UNTRANSLATED
					[translated_umc_strings]
						name = NAME
						description = DESCRIPTION
					[/translated_umc_strings]
				[/language]
			[/languages]
		[/umc]
	[/umc_list]
\end{lstlisting}

The fields in [umc] have the following meaning:
\begin{description}
\item[ID]
	See \cref{section:communication_protocol:server:umc.cfg}.

\item[NAME]
	See \cref{section:communication_protocol:server:umc.cfg}.

	The version in [language] is the translated name of the UMC.

\item[TYPE]
	See \cref{section:communication_protocol:server:umc.cfg}.

\item[DESCRIPTION]
	See \cref{section:communication_protocol:server:umc.cfg}.

\item[VERSION]
	See \cref{section:communication_protocol:server:umc.cfg}.

\item[SIZE]
	See \cref{section:communication_protocol:server:umc.cfg}.

\item[TIMESTAMP]
	The timestamp of the last upload. For the [umcs] it is the upload
	of the umc, for the [language] it is the last upload of the
	po-file. (NOTE MERGE WITH MASTER LIST.)

\item[DOWNLOADS]
	See \cref{section:communication_protocol:server:umc.cfg}.

\item[UPLOADS]
	See \cref{section:communication_protocol:server:umc.cfg}.

\item[LANGUAGE]
	The id of a language locale. The one in [umc] is the language of
	the UMC. If this language is empty American English is assumed. The UMC
	is only translatable if this is set to American English (this should be
	mentioned in the PBL section).
\end{description}


The [dependencies] lists the dependencies of the UMC, See
\cref{section:communication_protocol:server:umc.cfg}.

The [languages] contains the list of languages the addon is (partly)
translated to. The fields mean:
\begin{description}
\item[ID]
	The id of the language the translation for.

\item[TIMESTAMP]
	The timestamp the po-file was last updated. NOTE UNIX stamp, should be
	at other stamp as well.

\item[TRANSLATED]
	The number of translated strings, if this entry is $0$, then the entire
	section is not printed.

\item[FUZZY]
	The number of strings that are fuzzy.

\item[UNTRANSLATED]
	The number of strings that are not translated.

NOTE MAYBE ALSO ADD UP AND DOWNLOAD COUNTER and obviously the SIZE
\end{description}

The section [translated\_umc\_strings] duplicates some strings from the
general [umc] section. This version contains the translated string from
the po file. If not translated or fuzzy it contains the original, just like
normal with gettext. The keys in this section corrospondent with the ones in
[campaign].

\section{Request umc download}
\label{wire:request_umc_download}

This requests to download an umc.

Request:
\begin{lstlisting}
	[request_umc_download]
		id = ID
		[languages]?
			[language]*
				ID = LANGUAGE
			[/language]
		[/languages]
	[/requst_umc_download]
\end{lstlisting}

The fields are:
\begin{description}
\item[ID]
	The id of the umc to download.

\item[LANGUAGE]
	The id of the requested language to download.

\end{description}

Replay:
\begin{lstlisting}
	[umc_download]
		id = ID
		status = STATUS
		[languages]?
			[language]+
				id = LANGUAGE
				status = STATUS
			[/language]
		[/languages]
	[/umc_download]
\end{lstlisting}

This is a mirror of the request, only all items have a status field. The
status can be OK it the file exists and can be downloaded. Or ERROR if not
possible.

The files send are in the order of appearance in the request and reply. The
files are named:
\begin{description}
\item[ID.gz]
	The gzipped campaign files. The file is packed in a single config file
	to be extracted to the local file-system.

\item[ID-LANGUAGE.gz]
	A gzipped mo-file for a language. If gzip's compression ratio for mo
	files is too small the mo-file will be send directly.

\end{description}


\section{Request umc change password}
\label{wire:request_umc_change_password}

This request changes the password.


\section{Request umc upload}
\label{wire:request_umc_upload}


\section{Request umc delete}
\label{wire:request_umc_delete}

\section{Request license}
\label{wire:request_license}
